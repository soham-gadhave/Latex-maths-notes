\documentclass[a4paper, titlepage]{article}
\usepackage{bm}
\usepackage{amsthm}
\usepackage{amssymb}
\usepackage{enumitem}
\usepackage{nicefrac}
\usepackage{tabularx}
\usepackage[utf8]{inputenc}
\usepackage{amsmath, mathtools}
\usepackage[margin=1in]{geometry}

\newtheorem{theorem}{Theorem}[section]
\newtheorem{definition}{Definition}[section]
\newtheorem{corollary}{Corollary}[theorem]

\title{Linear Algebra}
\author{Soham Gadhave}
\date{$5^{\text{th}}$ September 2021}

\begin{document}

\maketitle

\section{Field}

Let F be non-empty set. Let '+' and '.' be two binary operations. Then, \\
$\implies (F, +)$ must be an Abelian group 
\begin{enumerate}[label=\alph*]
    \item \textbf{Closure:} For all $ x, y \in \text{F}, x + y = z \in \text{F}$ 
    \item \textbf{Associative:} For all $x, y, z \in $ F, $(x + y) + z = x + (y + z)$
    \item \textbf{Identity:} For all $x \in $ F, there exists an element $e \in $ F such that $x + e = e + x = x$
    \item \textbf{Inverse:} For all $x \in $ F, there exists an element $y \in $ F such that $x + y = e$
    \item \textbf{Commutative:} For all $x, y \in $ F, $x + y = y + x$
    \item[] First Four properties forms a group and all the five properties forms an Abelian group on '+'
\end{enumerate}
$\implies (F, *)$ must be an Abelian group 
\begin{enumerate}[label=\alph*]
    \item \textbf{Closure:} For all $ x, y \in \text{F}, x \cdot y = z \in \text{F}$ 
    \item \textbf{Associative:} For all $x, y, z \in $ F, $(x \cdot y) \cdot z = x \cdot (y \cdot z)$
    \item \textbf{Identity:} For all $x \in $ F, there exists an element $e \in $ F such that $x \cdot e = e \cdot x = x$
    \item \textbf{Inverse:} For every \textbf{non-zero} element $x \in $ F, there exists an element $y \in $ F such that $x \cdot y = e$
    \item \textbf{Commutative:} For all $x, y \in $ F, $x \cdot y = y \cdot x$
    \item[] First Four properties forms a group and all the five properties forms an Abelian group on '$\cdot$'
    \item[] \textbf{Note:} Identity is the property of the set F and
    Inverse is the property of the elements of the set F \\
\end{enumerate}
$ \implies $ \textbf{Distributve Laws:}
    \begin{enumerate}
        \item $ a\cdot(b + c) = a\cdot b + a\cdot c $
        \item $ (b + c)\cdot a = b\cdot a + c\cdot a $
    \end{enumerate}

\begin{center}
    \date{$6^{\text{th}}$ September 2021}
\end{center}

\section{Vector Space}

Let V be a non empty set and F be a field. Let '+' and '$\cdot$' 
be two binary operations.
\begin{description}
    \item[\textbf{Vector Addition:}] \hfill
    \begin{itemize}
        \item For all $ \vec{u}, \vec{v} \in $ V, $ \vec{u} + \vec{v} \in $ V. 
        \item For all $ \vec{u}, \vec{v}, \vec{w} \in $ V, $ (\vec{u} + \vec{v}) + \vec{w} = \vec{u} + (\vec{v} + \vec{w}) \in $ V. 
        \item For all $ \vec{u} \in $ V, there exists an $\vec{e} \in $ V such
        that $ \vec{u} + \vec{e} = \vec{u} = \vec{e} + \vec{u} $. 
        \item For all $ \vec{u} \in $ V, there exists a $ \vec{v} \in V $ 
        such that $ u + v = v + u = e $. 
        \item For all $ \vec{u}, \vec{v} \in $ V, $ \vec{u} + \vec{v} = \vec{v} + \vec{u} $. 
    \end{itemize}
    \item \textbf{Scalar Multiplication: } \hfill
    \begin{itemize}
        \item Let $ \vec{u} \in $ V and $ k \in $ F, then 
        $ ku \in $V.
        \item Let $ \vec u \in V $ and $ k_1, k_2 \in $ F, then 
        $ (k_1 + k_2)\vec{u} = k_1\vec{u} + k_2\vec{u} $ 
        \item Let $ \vec{u}, \vec{v} \in V $ and $ k \in $ F,
        then $ k(\vec u + \vec v) = k\vec u + k\vec v $
        \item Let $ k_1, k_2 \in $ F and $ \vec u \in $ V, then
        $ (k_1k_2)\vec u = k_1(k_2\vec u)  $ 
        \item There exists $ 1 \in F $ such that for all $ \vec u
        \in $ V, $ 1\cdot\vec u = \vec u$
    \end{itemize}
\end{description}
If a non empty set V satisfies all these properties over the Field
F under the binary operations '+' and '$\cdot$', then V is said to
be a Vector space over field F.
\begin{description}
    \item[\textbf{Notations:}] 
        \begin{itemize}
            \item[]
            \item $\mathbb{R} = (-\infty, \infty)$
            \item $\mathbb{R}^2 = \{ (x, y) \hspace{3pt} | \hspace{3pt} x, y \in \mathbb{R} \}$
            \item $\mathbb{R}^3 = \{ (x, y, z) \hspace{3pt} | \hspace{3pt} x, y, z \in \mathbb{R} \}$
            \item $\mathbb{R}^n = \{ (x_1, x_2, \cdots, x_n) \hspace{3pt} | \hspace{3pt} x_1, x_2, \cdots, x_n \in \mathbb{R} \}$
        \end{itemize}
    \item[]  
\end{description}



\section{Vector Subspaces}

% ================================================================

\begin{center}
    \date{$7^{\text{th}}$ September 2021}
\end{center}

\begin{definition}
    \label{subspacedef1}
    Let $V$ be a vector space over the field $F$ and $W$ be its subset. 
    $W$ is called the subspace of $V$ if W is itself a vector space 
    over the same field $F$ with the same binary operations as of V.
\end{definition}
\begin{center}
    OR
\end{center}
\begin{definition}
    \label{subspacedef2}
    Let $V$ be a vector space over the field $F$ and $W$ be its subset. 
    $W$ is called the subspace of $V$ over the same field $F$ with the 
    same binary operations if for all $\alpha, \beta \in F$ and u, v
    in $W$.
\end{definition}
\begin{align*}
    \alpha\cdot\bm{u} + \beta\cdot\bm{v} \in W
\end{align*}
\begin{center}
    OR
\end{center}
\begin{definition}
    \label{subspacedef3}
    Let $V$ be a vector space over the field $F$ and $W$ be its subset. 
    $W$ is called the subspace of $V$ over the same field $F$ with the 
    same binary operations if for all $u, v \in W$ and $\alpha \in F$.
\end{definition}
\[ \bm{u} + \bm{v} \in W \text{ and } \alpha \bm{u} \in W \]
\subsection{Examples}

\begin{enumerate}[label=\textbf{\arabic*.}]
    \item Let $V$ be a set of all functions from $S$ to $F$. Is $V$ a
    vector space under the binary operations
    $ (f + g)(x) = f(x) + g(x) $ and $ (cf)(x) = c\cdot f(x) $ where S is
    a non-empty set.
    \item[] $ V = \{ f \left. \right\vert f: S \rightarrow V \}$, $S \neq \phi$ \hfill \\ 
    \textbf{Vector Addition}:
    \begin{enumerate}[label=\textbf{\alph*}]
        \item \textbf{Closure}: 
        Let $f, g \in V \implies f: S \rightarrow F,\; g: S \rightarrow F$
        $$(f + g)(x) = f(x) + g(x) \; \forall x \in S$$
        and since 
        $$f(x), \; g(x) \in F, \; f(x) + g(x) \in F$$
        therefore 
        $$(f + g): S \rightarrow F \implies (f+g) \in V$$
        \item \textbf{Association: } For all $x \in S$
        \begin{align*}
            \left( (f + g) + h \right)(x) &= \left( f(x) + g(x) \right) + h(x) \\
                       &= f(x) + \left( g(x) +h(x) \right) \text{, since } f(x), \; g(x), \; h(x) \in F \\ 
                       &= f + (g + h)(x) \\
        \end{align*}
        \item \textbf{Identity: }
        For all $f$ we have to find an $e$ (if it exists) such that
        $$ f + e = f $$
        or
        $$ f(x) + e(x) = f(x), \; \forall x \in S $$
        which implies
        $$ e(x) = 0 \; \forall x \in S $$
        $e(x) = 0$ is the Identity element.
        \item \textbf{Inverse: } For every $f$ there must exist $g$
        such that 
        \begin{align*}
            f + g &= e \\
            \implies f(x) + g(x) &= e(x), \; \forall x \in S \\ 
            \implies g(x) &= 0 - f(x) \\ 
            \implies g(x) &= -f(x) 
        \end{align*}
        There exist $g = -f$ for every $f$ in $V$.
        \item \textbf{Commutative: } For all $x \in S$ 
        \begin{align*}
            (f + g)(x) &= f(x) + g(x) \\
                       &= g(x) + f(x) \text{, since } f(x), \; g(x) \in F \\ 
                       &= (g + f)(x) \\
        \end{align*}
    \end{enumerate}
    \textbf{Scalar Multiplication: }
        \begin{enumerate}[label=\textbf{\alph*}]
            \item Let $f \in V$ and $k \in F$, then as per definition, for $x \in S$ 
            $$
                (k \cdot f)(x) = k\cdot f(x)
            $$
            since $k, f(x) \in F $ therefore $k\cdot f(x) \in F$. This 
            implies $c\cdot f \in V$
            \item Let $f \in V$ and $k_1, k_2 \in F$, then for all $x$
            \begin{align*}
                ((k_1 + k_2)\cdot f)(x) &= (k_1 + k_2)\cdot f(x) \\
                                        &= k_1\cdot f(x) + k_2\cdot f(x) \\
                                        &= (k_1\cdot f)(x) + (k_2\cdot f)(x)
            \end{align*}
            Therefore $(k_1 + k_2)\cdot f = k_1\cdot f + k_2\cdot f$
            \item Let $f \in V$ and $k_1, k_2 \in F$, then for all $x$
            \begin{align*}
                ((k_1k_2)\cdot f)(x) &= (k_1k_2)\cdot f(x) \\
                                        &= k_1(k_2\cdot f(x)) \text{, since } k_1, k_2, f(x) \in F\\
                                        &= k_1(k_2\cdot f)(x)
            \end{align*}
            which implies $(k_1k_2)\cdot f = k_1(k_2\cdot f)$.
            \item Let $f, g \in V$ and $k \in F$, then for all $x$
            \begin{align*}
                (k\cdot (f+g))(x) &= k\cdot (f(x) + g(x)) \\
                                        &= k\cdot f(x) + k\cdot g(x) \text{, since } k, f(x), g(x) \in F\\
                                        &= (k\cdot f)(x) + (k\cdot g)(x)
            \end{align*}
            \item Let $f \in V$, $1 \in F$ then for all $x$ 
            $$ 1\cdot f(x) = f(x) $$
            which implies $1\cdot f = f$
        \end{enumerate}
        Hence $V$ is a vector space.
    \item Let V be the set of all polynomials from $F \rightarrow F$.
    Is $V$ a vector space over the field $F$?
    \item[] 
    $V = \{ a_0 + a_1x + a_2x^2 + \cdots + a_nx^n + \cdots \; | \; a_i \in F \; \forall i \} $ \hfill \\
    \textbf{Vector Addition: }
    \begin{itemize}
        \item Let $p, q \in V$, then $p + q \in V$
        \item Let $p, q, r \in V$, then $(p + q) + r = p + (q + r)$
        \item Let $ p \in V $, then for $q = \textbf{0} \in V$, $p + q = p$
        \item Let $ p \in V $, then for $q = -p \in V$, $p + q = 0$
        \item Let $p, q \in V$, then $p + q = q + p$
    \end{itemize}
    \textbf{Scalar Multiplication: }
    \begin{itemize}
        \item Let $p \in V, k \in F$, then $k\cdot f \in V$ 
        \item Let $p \in V$ and $k_ 1, k_2 \in F$, then $(k_1 + k_2)\cdot p = k_1\cdot p +  k_2\cdot p$
        \item Let $p \in V$ and $k_ 1, k_2 \in F$, then $(k_1k_2)\cdot p = k_1(k_2\cdot p)$
        \item Let $p, q \in V$ and $k \in F$, then $k\cdot (p + q) = k\cdot p +  k\cdot q$
        \item Let $p \in V$ and $1 \in F$, then $1\cdot p = p$
    \end{itemize}
    \item Let $V$ be the set of all polynomials of degree n. Is $V$ a vector space?
    \item[] $V$ is not a vector space because the following properties fails:
    \begin{itemize}
        \item \textbf{Closure: } Consider the polynomials 
        $p = x^n + x + 1$ and $q = -x^n + x + 1$, then
        $$ p + q = 2x + 1 $$
        which doesn't belong to V because it degree is not $n$.
        \item \textbf{Identity: } The degree of the polynomial $p = 0$
        is not defined, but it is certainly doesn't belong to $V$
    \end{itemize}
    \item $V$ is the set of all polynomials whose degree is atmost $n$.
    \item[]
    $V = \{ a_0 + a_1x + a_2x^2 + \cdots + a_kx^k \; | \; a_i \in F \; \forall i \; \text{and} \; k \le n \} $ \hfill \\ 
    It is a vector space over $F$ because it 
    satisfies all the properties of vector space and for that of the
    identity element \textbf{0}, it belongs to $V$ (if we put $a_i = 0$ for all $i$'s)
    \item $V$ is the set of all polynomials whose degree is atleast $n$.
    \item[] Same as \textbf{Ex \#3} 
    \item $V$ is the set of all $m$x$n$ matrices.
    \item[] Yes $V$ is a vector space
    \begin{center}
        \date{$9^{\text{th}}$ September}
    \end{center}
    \item $W_1 = \{ f \in V \vert \; f \text{ is a continous function} \}$ \hfill \\
    $W_1 \subseteq V$, where $V$ is the vector space of all functions $f: S \rightarrow F$.
    We know that \hfill \\
    $\textbf{0}(x) = 0$ for all $x \in S$, hence \textbf{0} in $W_1$. Now 
    for $f \text{ and } g \in W_1$ consider
    \[ (\alpha f + \beta g)(x), \text{ for all } x \in S \]
    This function is continous because it is the sum of two 
    constinous functions. Hence \( (\alpha f + \beta g)(x) \) is also
    in $W_1$. Therefore $W_1$ is a subspace
    \item $W_3 = \{ f \in V \left. \right\vert f(-x) = f(x) \quad \forall x \in S \} $ \hfill \\
    $ = $ set of all even functions \hfill \\
    $\textbf{0}(x) = 0$, for all $x \in S$, hence 
    $\textbf{0}(x) = \textbf{0}(-x)$ and therefore $\textbf{0} \in W_3$. Now Let
    \[ h(x) = (\alpha\cdot f + \beta\cdot g)(x) \quad \text{for all } x \in S\]
    Then
    \begin{align*}
        h(x) &= (\alpha\cdot f)(x) + (\beta\cdot g)(x) \quad \\
            &= \alpha f(x) + \beta g(x) \\
            &= \alpha f(-x) + \beta g(-x) \\
            &= (\alpha\cdot f)(-x) + (\beta\cdot g)(-x) \quad \\
            &= (\alpha\cdot f + \beta\cdot g)(-x) \quad \\
            &= h(-x)
    \end{align*}
    which implies $(\alpha\cdot f + \beta\cdot g)$ is an even function.
    Therefore $(\alpha\cdot f + \beta\cdot g) \in W_3$ and $W_3$ is a
    subspace.
    \item $W_4 = \{ f \in V \left. \right\vert f(x) = -f(-x) \quad \forall x \in S \} $ \hfill \\
    $ = $ set of all odd functions \hfill \\
    $\textbf{0}(x) = 0$, for all $x \in S$, hence 
    $\textbf{0}(x) = -\textbf{0}(-x)$ and therefore $\textbf{0} \in W_4$. Now Let
    \[ h(x) = (\alpha\cdot f + \beta\cdot g)(x) \quad \text{for all } x \in S\]
    Then
    \begin{align*}
        h(x) &= (\alpha\cdot f)(x) + (\beta\cdot g)(x) \quad \\
            &= \alpha f(x) + \beta g(x) \\
            &= \alpha f(-x) + \beta g(-x) \\
            &= -(\alpha\cdot f)(-x) - (\beta\cdot g)(-x) \quad \\
            &= -(\alpha\cdot f + \beta\cdot g)(-x) \quad \\
            &= -h(-x)
    \end{align*}
    which implies $(\alpha\cdot f + \beta\cdot g)$ is an odd function.
    Therefore $(\alpha\cdot f + \beta\cdot g) \in W_4$ and $W_4$ is a
    subspace 
    \item $V = \mathbb{R}^2, F = \mathbb{R}$ and 
    $$W = \{ (x, y) \vert x, y \in \mathbb{R} \text{ and } x \ge 0 \}$$
    $(0, 0) \in W$. Let u = (1, 1) and $\alpha = -1$, therefore 
    $\alpha\cdot u \notin W$ because $(-1, -1)$ has its $x$ component 
    is negative and $W$ is not a subspace.
    \item $W_5 = \{ A = M_{n\times n}(F) \left. \right\vert 
    a_{ij} = a_{ji} \text{ for all } i, j \}$ \hfill \\
    $W_5$ is the set of all $n\times n$ symmetric matrices, hence for
    all $A \in W_5,{} A = A^\intercal$. \hfill \\
    Null matrix = $\textbf{0} =  
    \begin{bmatrix}
        0 & 0 & \cdots & 0 \\
        0 & 0 & \cdots & 0 \\
        \vdots & \vdots & \ddots & \vdots \\ 
        0 & 0 & \cdots & 0
    \end{bmatrix}$. It is clear that 
    $\textbf{0} = \textbf{0}^\intercal$. Hence $\textbf{0} \in W_5$. \hfill \\
    Let 
    $C = \alpha A + \beta B$
    , where $A, B \in W_5$ and $\alpha, \beta \in F$, then 
    \begin{align*}
        C^\intercal &= (\alpha A + \beta B)^\intercal \\
                    &= (\alpha A)^\intercal + (\beta B)^\intercal \\
                    &= \alpha A^\intercal + \beta B^\intercal \\
                    &= \alpha A + \beta B \\
                    &= C
    \end{align*}
    which implies $C = C^\intercal$ for all $A, B \in W_6$ and 
    $\alpha, \beta \in F$, and hence $C \in W_6$. $W_6$ is a subspace
    \item $W_6 = \{ A = M_{n\times n}(F) \left. \right\vert 
    a_{ij} = -a_{ji} \text{ for all } i, j \}$ \hfill \\
    $W_6$ is the set of all $n\times n$ symmetric matrices, hence for
    all $A \in W_6,{} A = -A^\intercal$. \hfill \\
    Null matrix = $\textbf{0} =  
    \begin{bmatrix}
        0 & 0 & \cdots & 0 \\
        0 & 0 & \cdots & 0 \\
        \vdots & \vdots & \ddots & \vdots \\ 
        0 & 0 & \cdots & 0
    \end{bmatrix}$. It is clear that 
    $\textbf{0} = \textbf{0}^\intercal$. Hence $\textbf{0} \in W_6$. \hfill \\
    Let 
    $C = \alpha A + \beta B$
    , where $A, B \in W_6$ and $\alpha, \beta \in F$, then 
    \begin{align*}
        C^\intercal &= (\alpha A + \beta B)^\intercal \\
                    &= (\alpha A)^\intercal + (\beta B)^\intercal \\
                    &= -\alpha A^\intercal - \beta B^\intercal \\
                    &= -(\alpha A + \beta B) \\
                    &= -C
    \end{align*}
    which implies $C = -C^\intercal$ for all $A, B \in W_6$ and 
    $\alpha, \beta \in F$, and hence $C \in W_6$. $W_6$ is a subspace
    \item $W_7 = \{ A = M_{n\times n}(F) \left. \right\vert 
    A = A^\theta \}$ \hfill \\
    Null matrix = $\textbf{0} =  
    \begin{bmatrix}
        0 & 0 & \cdots & 0 \\
        0 & 0 & \cdots & 0 \\
        \vdots & \vdots & \ddots & \vdots \\ 
        0 & 0 & \cdots & 0
    \end{bmatrix}$. It is clear that 
    $\textbf{0} = \textbf{0}^\theta$. Hence $\textbf{0} \in W_7$. \hfill \\
    Let 
    $C = \alpha A + \beta B$
    , where $A, B \in W_7$ and $\alpha, \beta \in F$, then 
    \begin{align*}
        C^\theta &= (\alpha A + \beta B)^\theta \\
                 &= (\overline{\alpha A + \beta B})^\top \\
                 &= (\overline{\alpha A})^\intercal + (\overline{\beta B})^\intercal \\
                 &= \overline{\alpha} \overline{A}^\intercal + \overline{\beta B}^\intercal \\
                 &= \overline{\alpha} A^\theta + \overline{\beta} B^\theta \\
                 &= \overline{\alpha} A + \overline{\beta} B
    \end{align*}
    which is not equal to $C$ for all $\alpha, \beta \in F$. 
    For $F = \mathbb{R}, \; W_7$ is vector subpace but for 
    $F = \mathbb{C}, \; W_7$ is not a subspace
    \item $W_7 = \{ A = M_{n\times n}(F) \left. \right\vert 
    A = -A^\theta \}$ \hfill \\
    Null matrix = $\textbf{0} =  
    \begin{bmatrix}
        0 & 0 & \cdots & 0 \\
        0 & 0 & \cdots & 0 \\
        \vdots & \vdots & \ddots & \vdots \\ 
        0 & 0 & \cdots & 0
    \end{bmatrix}$. It is clear that 
    $\textbf{0} = -\textbf{0}^\theta$. Hence $\textbf{0} \in W_7$. \hfill \\
    Let 
    $C = \alpha A + \beta B$
    , where $A, B \in W_7$ and $\alpha, \beta \in F$, then 
    \begin{align*}
        C^\theta &= (\alpha A + \beta B)^\theta \\
                 &= (\overline{\alpha A + \beta B})^\top \\
                 &= (\overline{\alpha A})^\intercal + (\overline{\beta B})^\intercal \\
                 &= \overline{\alpha} \overline{A}^\intercal + \overline{\beta B}^\intercal \\
                 &= \overline{\alpha} A^\theta + \overline{\beta} B^\theta \\
                 &= - (\overline{\alpha} A + \overline{\beta} B)
    \end{align*}
    which is not equal to $C$ for all $\alpha, \beta \in F$. 
    For $F = \mathbb{R}, \; W_7$ is vector subpace but for 
    $F = \mathbb{C}, \; W_7$ is not a subspace
\end{enumerate}

% ================================================================

\begin{center}
    \date{$10^{\text{th}}$ September 2021}
\end{center}

\subsection{Intersection of two subspaces}

\begin{theorem}
    Let $V$ be a vector space over the field F and $W_1$ and $W_2$ 
    be two subspaces of the vector space $V$. Then $W = W_1 \cap W_2$ 
    is also a subspace of $V$.
    \begin{proof}
        $W_1$ and $W_2$ are subspaces of $W$. Hence
        \begin{align*}
            & \bm{0} \in W_1 \text{ and } \bm{0} \in W_2 \\
            \implies & \bm{0} \in W_1 \cap W_2 \\
            \implies & W = (W_1 \cap W_2) \neq \bm{\phi}
        \end{align*}
        Let $\alpha, \beta \in F$ and $u, v \in W$. Since $u, v \in W$,
        then $u, v \in W_1$ and $u, v \in W_2$ which implies 
        $\alpha u + \beta v \in W_1$ and $\alpha u + \beta v \in W_2$,
        because $W_1$ and $W_2$ are subspaces. 
        Since $\alpha u + \beta v \in W_1$ and $\alpha u + \beta v \in W_2$, 
        $\alpha u + \beta v \in W$ for all $\alpha, \beta \in F$ and 
        hence $W = W_1 \cap W_2$ is a subspace.
    \end{proof}
\end{theorem}
\begin{corollary}
    Let $V$ be a vector space over the field F and $W_i$ the be 
    arbitrary subspaces of the vector space $V$ where $i \in \mathbb{N}$. 
    Then $W = \displaystyle\bigcap_{i = 1}^{\infty} W_i$ is also a 
    subspace of $V$.
\end{corollary}
\subsection{Union of two subspaces}
    \begin{theorem}
        Union of two subspaces of the same vector space is a 
        subspace if and only if one of them is contained in another.
    \end{theorem}
    \begin{proof}
        Let $V$ be a vector space over the field F and $W_1$ and $W_2$ 
        be two subspaces of the vector space $V$.
        \begin{itemize}
            \item \textbf{Forward: } $W = W_1 \cup W_2$ is a 
            subspace (Hypothesis) \hfill \\
            (Proof by contradiction) Let us assume that one doesn't 
            contain other ($W_1 \not\subseteq W_2$ and 
            $W_2 \not\subseteq W_1$). \\
            \begin{tabularx}{\linewidth}{c c c}
                $W_1 \not\subseteq W_2$
                &
                and
                &
                $W_2 \not\subseteq W_1$ \\
                \begin{minipage}[c]{0.45\linewidth}
                    There exists a $x \in W_1$ such that $x \notin W_2$,
                    which implies $x \in W = W_1 \cup W_2$
                \end{minipage}
                &
                &
                \begin{minipage}[c]{0.45\linewidth}
                    There exists a $y \in W_2$ such that $y \notin W_1$,
                    which implies $y \in W = W_1 \cup W_2$
                \end{minipage}
            \end{tabularx}
            \[
                \because W \text{ is a subspace } \therefore x + y
                \in W
            \]
            \[
                \because x + y \in W, \text{ then either }
            \]
            \begin{tabularx}{\linewidth}{c c c}
                $x + y \in W_1$ and $x \in W_1$
                &
                or
                &
                $x + y \in W_2$ and $y \in W_2$ \\
                \begin{minipage}[c]{0.45\linewidth}
                    $\because W_1$ is a subspace $(x + y) - x \in W_1$, 
                    that is $y \in W_1$, which is a contradiction because
                    $y \in W_2$ and $y \notin W_1$.
                \end{minipage}
                &
                &
                \begin{minipage}[c]{0.45\linewidth}
                    $\because W_2$ is a subspace $(x + y) - y \in W_2$, 
                    that is $x \in W_2$, which is a contradiction because
                    $x \in W_1$ and $x \notin W_2$.
                \end{minipage} \\ \\
            \end{tabularx}
            Hence our assumption that one subspace doesnt contain
            other is false.
            \item \textbf{Backward: } $W_1 \subseteq W_2$ or 
            $W_2 \subseteq W_1$ (Hypothesis) \hfill \\
            \begin{tabularx}{\linewidth}{c c c}
                $W_1 \subseteq W_2$
                &
                or
                &
                $W_2 \subseteq W_1$ \\
                \begin{minipage}[c]{0.45\linewidth}
                    $W = W_1 \cup W_2 = W_2 \; (\because W_1 \subseteq W_2)$, 
                    hence W is subspace
                \end{minipage}
                &
                &
                \begin{minipage}[c]{0.45\linewidth}
                    $W = W_1 \cup W_2 = W_1 \; (\because W_2 \subseteq W_1)$, 
                    hence W is subspace
                \end{minipage}
            \end{tabularx}
        \end{itemize}
    \end{proof}

    \subsection{Sum of two subspaces}
    
    \begin{theorem}
        Let $V$ be a vector space over the field F and $W_1$ and $W_2$ 
        be two subspaces of the vector space $V$. Then $W = W_1 + W_2$ 
        is also a subspace of $V$. 
    \end{theorem}
    \begin{proof}
        $W = W_1 + W_2 = \{ w_1 + w_2 \; \left. \right\vert \; w_1 \in W_1 \text{ and } w_2 \in W_2 \}$. 
        $\bm{0} \in W_1$ and $W_2$, hence $\bm{0} + \bm{0} = \bm{0}$
        which also belongs to $W \neq \bm{\phi}$. For all $\alpha, 
        \beta \in F$ and $x, y \in W$ consider
        \[
            \alpha\cdot x + \beta\cdot y = \alpha\cdot(u + v) + \beta\cdot(p + q)   
        \]
        where $x = u + v$ for some $u \in W_1$ and $v \in W_2$ and
        $y = p + q$ for some $p \in W_1$ and $q \in W_2$. Therefore
        \[            
            \alpha\cdot x + \beta\cdot y = (\alpha u + \beta p) + 
            (\alpha v + \alpha q) \\
        \]
        since $u, p \in W_1$ and $v, q \in W_2$
        \[
            \alpha x + \beta y = s + t  
        \]
        where $s \in W_1$ and $t \in W_2$ and therefore 
        $\alpha x + \beta y \in W$ for all $\alpha, 
        \beta \in F$ and $x, y \in W$. Hence $W$ is subspace of $V$
    \end{proof}

\section{Linear Dependency and Independency}

    \subsection{Linear Combination}
    \begin{definition}
        Let $v$ be a vector of the vector space $V$. Then $v$ is a linear
        combination of $u_1, u_2 \in V$ if there exist $c_1, c_2 \in F$
        such that 
        \[ \boxed{v = c_1u_1 + c_2u_2} \]
        \textup{Generalized form: } For $u_1, u_2, \cdots, u_n \in V$
        $v \in V$ is Linear Combination of $u_1, u_2, \cdots, u_n$ if
        there exist $c_1, c_2 \cdots, c_n \in F$ such that
        \[ \boxed{v = c_1u_1 + c_2u_2 + \cdots c_nu_n} \]
    \end{definition}

    \begin{theorem}
        \label{lctheorem1}
        Set of all possible linear combinations of 
        $u_1, u_2, \cdots, u_n \in V$, forms a subspace of $V$ over 
        the field $F$
    \end{theorem}
    \begin{proof}
        The set $S = \{ c_1u_1 + c_2u_2 + \cdots + c_nu_n \in V \; \vert \; c_i \in F, \forall i's \}
        = \left\{ \displaystyle\sum_i^n c_iu_i \left. \right\vert c_i \in F, \forall i's \right\} $. \hfill \\
        Putting $c_i = 0$ for all $i$, we get the $\bm{0}$ vector, implying
        $\bm{0} \in S$. Consider $u, v \in S$, then there exists 
        $\alpha_1, \alpha_2, \cdots, \alpha_n$ and $\beta_1, \beta_2, 
        \cdots \beta_n \in F$ such that
        \begin{align*}
            u &= \alpha_1u_1 + \alpha_2u_2 + \cdots + \alpha_nu_n \\
            v &= \beta_1u_1 + \beta_2u_2 + \cdots + \beta_nu_n
        \end{align*}
        Then
        \begin{align*}
            u + v &= \alpha_1u_1 + \alpha_2u_2 + \cdots + \alpha_nu_n
                    +
                    \beta_1u_1 + \beta_2u_2 + \cdots + \beta_nu_n \\
                  &= (\alpha_1 + \beta_1)u_1 + (\alpha_2 + \beta_2)u_2
                     + \cdots +
                     (\alpha_n + \beta_n)u_n
        \end{align*}
        since $\alpha_i, \beta_i \in F$ for all $i$'s $\implies$ 
        $\alpha_i + \beta_i \in F$ for all $i$'s and hence $u + v
        \in S$. \\
        Let $\alpha \in F$, then 
        \begin{align*}
            \alpha u &= \alpha(\alpha_1u_1 + \alpha_2u_2 + \cdots + \alpha_nu_n) \\
                     &= (\alpha\alpha_1)u_1 + (\alpha\alpha_2)u_2 + \cdots + (\alpha\alpha_n)u_n
        \end{align*}
        also belongs in S because for all $\alpha_1, \alpha_2, \cdots, 
        \alpha_n \text{ and } \alpha \in F, \; \alpha\alpha_i \in F$ 
        for all $i$'s. \\
        Therefore S is subspace by Definition (\ref{subspacedef3}).
    \end{proof}

    \subsection{Span}
    \begin{definition}
        \label{span1}
        Let $V$ be a vector space over the field $F$and $S \subseteq V$.
        Then the set of all possible linear combinations of the set
        $S$ is called the span of that set or the spanning set of $S$ 
        and is written as $\textup{span(S)}$.
        \begin{align*}
            S &= \{ v_1, v_2, \cdots, v_n \} \\
            \textup{span}(S) &= \{ c_1v_1 + c_2v_2 + \cdots + c_nv_n \vert c_i \in F, v_i \in S \} \\
                           &= \left. \left\{ \displaystyle\sum_{i=1}^n c_iv_i \; \right\vert \; c_i \in F, v_i \in S \right\}
        \end{align*}
        \begin{center}
            OR
        \end{center}
    \end{definition}
    \begin{definition}
        \label{span2}
        Span $S$ is the intersection of all subspaces of $V$
        containing $S$.
    \end{definition}
    \begin{corollary}
        Let $V$ be a vector space over the field $F$ and 
        $S \subseteq V$, then$\textup{ Span}(S)$ is a subspace of $V$.
    \end{corollary}
    \begin{proof}
        For $S = \bm{\phi}$, from definition \textbf{(\ref{span2})} 
        $S$ is subspace of $V$ and for $S \neq \bm{\phi}$, from
        Definition \textbf{(\ref{span1})} and Theorem \textbf{(\ref{lctheorem1})}
        $S$ is a subspace of $V$.
    \end{proof}
    
    \subsection{Linearly Independent (L.I) and Linearly Dependent (L.D)}

    \begin{definition}
        Let $V$ be a vector space and $S = \{ v_1, v_2, \cdots,  v_n \}$ be a subset of $V$. 
        Then the set S is \textbf{linearly independent} if
        \[ 
            c_1v_1 + c_2v_2 + \cdots + c_nv_n = 0 \implies c_i = 0 
            \textup{ for all } i\textup{'s}    
        \]
        where $v_1, v_2, \cdots, v_n \in S$ and $c_1, c_2 \cdots, 
        c_n \in F$. If it is not linearly independent then its 
        \textbf{linearly dependent}, i.e there exists atleast one 
        $c_i \neq 0$ when $c_1v_1 + c_2v_2 + \cdots + c_nv_n = 0$.
    \end{definition}
    \begin{theorem}
        Let V be a vector space over the field F and S be a non
        empty subset of V. S is linearly dependent if and only if
        atleast one vector of $S$ can be written as a linear combination
        of others.
    \end{theorem}
    \begin{proof}
        Let $S = \{ v_1, v_2, \cdots v_n \}$ be non-empty subset of $V$. \hfill \\
        \textbf{Forward: }$S$ is linearly dependent. There exists a $c_i \neq 0$
        such that \[ c_1v_1 + \cdots + c_{j-1}v_{j-1} + c_jv_j + c_{j+1}v_{j+1} + \cdots + c_nv_n = 0 \]
        Without loss of generality we assume $c_j \neq 0$.
        \begin{align*}
            \therefore \; & c_jv_j = - c_1v_1 - \cdots - c_{j-1}v_{j-1} - c_{j+1}v_{j+1} - \cdots - c_nv_n \\
            & v_j = - \dfrac{c_1}{c_j}v_1 - \cdots - \dfrac{c_{j-1}}{c_j}v_{j-1} - \dfrac{c_{j+1}}{c_j}v_{j+1} - \cdots - \dfrac{c_n}{c_j}v_n \\
            & v_j = k_1v_1 + \cdots + k_{j-1}v_{j-1} + k_{j+1}v_{j+1} + \cdots + k_nv_n
    \end{align*}
    which says that $v_j$ is the linear combination of the remaining 
    vectors of $S$. \hfill \\
    \textbf{Backward: }Atleast one vector of S can be written as 
    remaining others. This implies that 
    \begin{align*}
        & v_i = c_1v_1 + \cdots + c_{i-1}v_{i-1} + c_{i+1}v_{i+1} + \cdots + c_nv_n \\
        \implies &c_1v_1 + \cdots + c_{i-1}v_{i-1} + (-1)v_i + c_{i+1}v_{i+1} + \cdots + c_nv_n = 0
    \end{align*}
    Let $c_i = -1$, then the linear combination of the vectors of $S$
    is 0 for atleast one $c_j \neq 0, (c_i = -1)$ and hence $S$ is linearly dependent.
    \end{proof}

    \subsubsection{Properties}
    \begin{enumerate}
        \item $S = \bm{\phi}$ is a Linearly Independent set.
        \item $S = \{ v \}$ is a singleton set, then 
        \begin{enumerate}
            \item If $v \neq 0$, then $c\cdot v \implies c = 0$.
            \item If $v = 0$, then $c\cdot v \implies c = 0$ for 
            all values of $c$.
        \end{enumerate}
        Hence S is linearly independent if $v \neq 0$ else linearly
        dependent.
        \item If $S$ is linearly dependent set in $V$, then subset of $S$
        may or may not be linearly dependent. \\
        \textbf{Ex. }$S = \{ (0,1), (1,0), (1, 1), (2, 2) \} \hfill \\$
        Consider $W_1 = \{ (0,1), (1,0) \} \subseteq S$ and $W_2 = \{ (1,1), (2,2) \} \subseteq S$.
        $W_1$ is linearly independent and $W_2$ is linearly dependent
        \item If $S$ is linearly dependent set in $V$, then superset
        of $S$ is also linearly dependent.
        \begin{proof}
            Suppose $S = \{ v_1, v_2, \cdots, v_m \}$, and 
            $W = \{ v_1, \cdots, v_m, v_{m+1} \cdots, v_n \}$ \\
            Clearly $W \supseteq S$. Now consider the linear combination
            of the vectors in $W$.
            \[ \underbrace{c_1v_1 + c_2v_2 + \cdots + c_mv_m}_{\text{ = 0 for atleast one $c_j \neq 0$}} + c_{m+1}v_{m+1} + c_{m+2}v_{m+2} +\cdots + c_nv_n  \]
            Therefore
            \[ c_1v_1 + c_2v_2 + \cdots + c_mv_m + c_{m+1}v_{m+1} + c_{m+2}v_{m+2} +\cdots + c_nv_n = 0 \]
            implies there exists atleast one $c_i = c_j \neq 0$.
            Hence $W \supseteq S$ is linearly dependent.
        \end{proof}
        \item If $S$ is a linearly independent set in $V$ then any
        subset of $S$ is also linearly independent.
        \begin{proof}
            \textbf{(Proof by contradiction)} Let us assume that a
            subset $T$ of $S$ is linearly dependent, then the linear
            combination 
            \begin{equation}
                c_1v_1 + c_2v_2 + \cdots + c_mv_m = 0
                \implies \text{that there exist atleast one } c_i \neq 0, 1 \le i \le m
            \end{equation}
            Now consider the original set $S$ which is linearly independent.
            The equation \[ c_1v_1 + c_2v_2 + \cdots + c_nv_n = 0 \]
            implies that $c_i = 0$ for all $i$'s. Lets inspect the
            linear combination more carefully.
            \begin{align*}
                & \underbrace{c_1v_1 + c_2v_2 + \cdots + c_mv_m}_{\text{ $= 0 \implies$ there exist a $c_j \neq 0$ }} 
                + \, \underbrace{c_{m+1}v_{m+1} + c_{m+2}v_{m+2} + 
                \cdots + c_nv_n}_{\text{ = 0 if we put $c_i = 0$ for all $i > n$  }} \\
                & = 0
            \end{align*}
            Hence
            \[ c_1v_1 + c_2v_2 + \cdots + c_mv_m + c_{m+1}v_{m+1} + c_{m+2}v_{m+2} + \cdots + c_nv_n = 0 \]
            implies there exist a $c_j \neq 0$. This shows that
            $S$ is a linearly dependent set which contradicts with
            our assumption and proves $T$ is a linearly independent
            set.
        \end{proof}
        \item If $S$ is a linearly independent set in $V$, then the
        superset of $S$ may or may not be linearly independent. \hfill \\
        \textbf{Ex. }$S = \{ (-1,0) \}$ and $T_1 = \{ (-1, 0), (0, 1) \}$
        and $T_1 = \{ (-1, 0), (-2, 0) \}$ then, 
        $T_1$ is linearly independent and $T_2$ is linearly dependent
    \end{enumerate}

\section{Basis}
    \begin{itemize}[label=$\implies$]
        \item span $S$ is the subspace of $V$
        \item span $\bm{\phi} = {0}$
        \item span $S$ is the intersection of all subspace of $V$ 
        containing $S$
        \item If $S$ is a subspace of $V$, then span $S = S$
        \item span(span $S$) $= $ span $S$, span(span($\cdots$ span($S$)$\cdots$)) $=$ span $S$
        \item If $S$ is a subspace, then span(span($\cdots$ span($S$)$\cdots$)) $=$ span $S$
    \end{itemize}
    \begin{definition}
        Let $V$ be a vector space over the field $F$ and $B$ be non 
        empty subset of $V$. Then $B$ is called the basis of $V$ if
        \begin{enumerate}[label=\textup{(\alph*)}]
            \item B is linearly independent
            \item \textup{span($B$)} = $V$
        \end{enumerate}
    \end{definition}
    \textbf{Dimension of a Basis: }The number of elements in the 
    basis is called the dimension of that basis.
    \begin{center}
        OR
    \end{center}
    The cardinality of the basis is called its dimension. \hfill \\
    $C(B) < \infty \implies V$ is a finite dimensional vector space \hfill \\
    $C(B) \nless \infty \implies V$ is an infinite dimensional vector space \hfill \\
    Number of Basis of any vector space can be infinite but dimension of a 
    vector space is unique.

    \subsection{Properties}
        Let $S$ be a subset of $V$ and span($S$) = V
        \begin{enumerate}
            \item For any $w \in V$, $S = \{ u_1, u_2, \cdots, u_n \}$
            then $S \cup \{ w \}$ also spans $V$, i.e span($S \cup \{ w \}$) $= V$
            \item If any $u_i$ is a linear combination of $u_1, u_2,
            \cdots, u_{i-1}$ then $S \setminus \{u_i\}$ also spans $V$, i.e
            $$\text{span}\left( S\setminus \{ u_i \} \right) = V$$
            \item Standard basis of the Euclidean space $\mathbb{R}^n$
            is \[
                    \{
                    (1, 0 , \cdots, 0),
                    (0, 1 , \cdots, 0),
                    \cdots,
                    (0, 0 , \cdots, 1)
                    \}
            \]
            \item $P[x] = \{ a_0, a_1x + \cdots + a_nx_n + \cdots \; \vert \; a_i \in F \text{ for all }i \}$.
            Then $$B = \{ 1, x, x^2, \cdots, x^n, \cdots \}$$
            is the basis for $P[x]$
            \item Let W be the set of all solution of the second order differential
            equation 
            \begin{equation}
                \label{2nddeglinearde}
                \dfrac{d^2y}{dx^2} + P(x)\dfrac{dy}{dx} + Q(x)y = 0
            \end{equation}
            Then W is forms a subspace.
            \begin{proof}
                $$W = \left\{ y(x) \; \middle| \; \dfrac{d^2y}{dx^2} + P(x)\dfrac{dy}{dx} + Q(x)y = 0 \right\}$$
                $y(x) = 0$ satisfies equation (\ref{2nddeglinearde}). Hence
                $y(x) = 0$ is a solution of the differential equation.
                Let $y_1(x)$ and $y_2(x)$ be two solutions of (\ref{2nddeglinearde}),
                then consider $\alpha y_1 + \beta y_2$.
                \begin{align*}
                    &\dfrac{d^2(\alpha y_1 + \beta y_2)}{dx^2} + P(x)\dfrac{d(\alpha y_1 + \beta y_2)}{dx} + Q(x)(\alpha y_1 + \beta y_2) \\
                    &= \left( \alpha\dfrac{d^2y_1}{dx^2} + \alpha P(x)\dfrac{dy_1}{dx} + \alpha Q(x)y_1 \right)
                    +  \left( \beta\dfrac{d^2y_2}{dx^2} + \beta P(x)\dfrac{dy_2}{dx} + \beta Q(x)y_2 \right) \\
                    &= \alpha\left( \dfrac{d^2y_1}{dx^2} + P(x)\dfrac{dy_1}{dx} + Q(x)y_1 \right)
                    +  \beta\left( \dfrac{d^2y_2}{dx^2} + P(x)\dfrac{dy_2}{dx} + Q(x)y_2 \right) \\
                    &= 0 + 0 = 0
                \end{align*}
                Hence W is a subspace
                The general solution $y$ of the above differential equation
                is given by 
                \begin{equation}
                    \label{2nddeggensoln}
                    y(x) = c_1y_1(x) + c_2y_2(x)
                \end{equation} where $y_1(x)$ and
                $y_2(x)$ are two solutions of (\ref{2nddeglinearde}).
                From (\ref{2nddeggensoln}) we can say that any solution
                can be written as the linear combination of $y_1$ and $y_2$.
                Hence $y_1$ and $y_2$ spans $W$ and since (\ref{2nddeggensoln})
                is the general solution $y_1$ and $y_2$ are linearly independent.
                Therefore $B = \{ y_1, y_2 \}$ is the basis for $W$ whose
                dimension is 2. \hfill \\
                \textbf{Generalization: }Let W be the set of all solutions
                of the differential equation
                \begin{equation}
                    \label{nthdeglinearde}
                    a_n\dfrac{d^ny}{dx^n} + a_{n-1}(x)\dfrac{dy}{dx} + \cdots + a_0(x)y = 0
                \end{equation}
                Then $W$ forms a subspace(vector space?) and its basis is
                given by $B = \{ y_1, y_2, \cdots, y_n \}$ with dimension
                equal to the order of the D.E $= n$ and where $y_1, y_2, \cdots, y_n$ 
                are the solutions of (\ref{nthdeglinearde}) and they form the 
                general solution of (\ref{nthdeglinearde}), that
                is \[ y = c_1y_1 + c_2y_2 + \cdots + c_ny_n \].
            \end{proof}
        \end{enumerate}
        \begin{theorem}
            Let V be finite dimensional vector space over the field F
            and let L be the set of linearly independent vectors in
            V and S be spanning set of V, then cardinality of L is 
            less than or equal to cardinality of \textup{span($S$)}.
        \end{theorem}
        \begin{proof}
            span($S$) = $V$ and $S$ can be either linearly independent or
            linearly dependent. If $S$ is linearly independent then by definition
            $S$ becomes the basis for $V$. If S is not linearly independent
            then its linearly dependent and atleast one vector of $S$ can
            be written as linear combination of other vectors of $S$ say
            $v_1$. Since $S$ spans $V$, any vector $v \in V$ can be written
            as linear combination of vectors in S.
            \begin{align*}
                v &= c_1v_1 + c_2v_2 + \cdots + c_nv_n \\
                &= c_1(k_1v_2 + k_2v_3 + \cdots + k_{n-1}v_n) + c_2v_2 + \cdots + c_nv_n \\
                &= (c_1k_1 + c_2)v_2 + (c_1k_2 + c_3)v_3 + \cdots + (c_1k_{n-1} + c_n)v_n \\
                &= \alpha_2v_2 + \alpha_3v_3 + \cdots + \alpha_nv_n \\
            \end{align*}
            which implies there exists $\alpha_i$ for every $v \in V$ such
            that $v$ can be written as a linear combination of vectors in
            $S$. Hence span($S$) $= V$        
        \end{proof}

        \begin{theorem}
            Let $V$ be a \textbf{finite dimensional vector space}, with
            dimension of V equal to $\bm{n}$, then any subset of $V$ with 
            $\bm{(n+1)}$ or more vectors is a linearly dependent set 
            (it cannot form a basis for $V$).
        \end{theorem}
        
        \setlength{\fboxsep}{1em}
            \noindent\fbox{
                \begin{minipage}[c]{0.935\linewidth}
                    \textbf{Notes: }
                    
                    \noindent Let $V$ be a \textbf{finite 
                    dimensional vector space}, with dim $V = n$ and $B$ be
                    a subset of $V$ with $\bm{n}$ number of vectors, then

                    If span($B$) = $V$, then $B$ is a basis of $V$.
                    \begin{center}
                        OR
                    \end{center}
                    If $B$ is linearly independent, then $B$ is a basis of $V$

                    \textbf{That is for checking whether a subset $B$ of $V$ with $n$ 
                    vectors is a basis of $V$ either check if it is Linearly
                    Independent or if it spans $V$}

                    Also the vector space $V = \{0\}$ has the set $B = \bm{\phi}$
                    as its basis

                \end{minipage}
            }

        \begin{theorem}
            Let $V$ be a \textbf{finite dimensional vector space} over 
            the field $F$ and $B$ be a subset of $V$. The set $B$ is a
            basis for $V$ if and only if every vector of $V$ can be written
            as a unique linear combination of the vectors in $B$.
        \end{theorem}
        \begin{proof}
            \textbf{Forward direction: }$B = \{ u_1, u_2, \cdots, 
            u_n \}$ is a basis for $V$.

            \noindent Let's assume that there exist two distinct set of scalars
            $\{ c_1, c_2, \cdots, c_n \}$ and $\{ k_1, k_2, \cdots, k_n \}$ 
            for any vector $v \in V$ such that
            $$ v = c_1u_1 + c_2u_2 + \cdots + c_nu_n $$
            $$ v = k_1u_1 + k_2u_2 + \cdots + k_nu_n .$$
            Substracting one from another, we get
            $$ (c_1-k_1)u_1 + (c_2-k_2)u_2 + \cdots + (c_n-k_n)u_n = 0$$ 
            since the vectors from the basis are all linearly 
            independent we have $c_i = k_i$ for all $i$'s, which is a 
            contradiction because they are suppose to be different 
            from our assumption.
            
            \noindent\textbf{Backward direction: }every vector of 
            $V$ can be written as a unique linear combination of 
            the vectors in $B$.
            
            \noindent There exists a unique set of scalars 
            $c_1, c_2, \cdots, c_n$ for every vector $v \in V$ 
            such that $$ v = c_1u_1 + c_2u_2 + \cdots + c_nu_n $$
            That is $B$ spans $V$. Now we have to show that the 
            vectors in $B$ are linearly independent, for that consider
            the $0$ vector. $$ 0 = 0\cdot u_1 + 0\cdot u_2 + \cdots + 0\cdot u_n. $$
            Since this combination is unique, there is no other way
            of writing the $0$ vector. This states that the equation
            $$ c_1u_1 + c_2u_2 + \cdots + c_nu_n = 0 $$ has only one
            solution and that is $c_i = 0$ for all $i$'s. Hence $B$ is
            a basis for $V$

        \end{proof}

    \subsection{How to check whether a set is a basis?}
    
        Suppose dimension of $V$ is $n$ and $S$ is a subset of $V$
        to be checked for basis.
        \begin{center}
            dim $V = n$ and $S \subseteq V$
        \end{center}

        \begin{minipage}[c]{0.9\linewidth}
            \begin{tabularx}{\linewidth}{p{0.33\linewidth} p{0.33\linewidth} p{0.33\linewidth}}
                If $\bm{|S| < n}$, then $S$ can never span $V$ and hence $S$ 
                is not a basis of $V$
                &
                If $\bm{|S| = n}$, then check either $S$ is linearly independent 
                or span($S$) $= V$ for it to be a basis of $V$
                &
                If $\bm{|S| > n}$, then $S$ becomes L.D and hence $S$ cannot form 
                a basis of $V$
            \end{tabularx}
        \end{minipage}

        \begin{theorem}
            Let $V$ be a finite dimensional vector space over the field
            $F$, and $W$ be its subspace, then $\textbf{\textup{dim}
            W} \le \textbf{\textup{dim} V}$
        \end{theorem}
    
    \subsection{Ordered basis and Direct Sum}
        \begin{description}
            \item[Coordinate vector:] Let V be a finite dimensional
            vector space over the field $F$ and B be a basis of V
            ($B = \{ v_1, v_2, \cdots, v_n \} $). Then for any 
            $v \in V$, $$v = c_1v_1 + c_2v_2 + \cdots + c_nv_n$$
            and $$
                    \bm{[v]}_B = \begin{bmatrix}
                                c_1 \\ c_2 \\ \vdots \\ c_n
                            \end{bmatrix}
                $$
            where $\bm{[v]}_B$ is called the coordinate vector of $v$ 
            w.r.t the basis $B$.
            \item[Direct Sum:] Let $W_1$ and $W_2$ be two sub-spaces
            of a vector space $V$, then $W_1 + W_2$ is called the 
            direct sum of $V$ if every member of $V$ can be uniquely 
            expressed as $w_1 + w_2$, where $w_1 \in W_1$ and 
            $w_2 \in W_2$. The direct sum of $V$ is denoted by
            $W_1 \bigoplus W_2$
            \item[Examples of finding the basis: ]
        \end{description}

        \begin{theorem}
            Let $W_1$ and $W_2$ be two subspaces of $V$, then 
            $W_1 + W_2$ is said to be the direct sum of $V$ if and 
            only if $W_1 \cap W_2 = \{ 0 \}$
        \end{theorem}
        
        \noindent \textbf{Example:} Let $V = \mathbb{R}^3$ and $W_1 =
        \{ (0, y, z) : y, z \in \mathbb{R} \}$ and $W_2 = \{ (x, y, 0) 
        : x, y \in \mathbb{R} \}$. Is $V$ the direct sum of $W_1$ 
        and $W_2$.

        \noindent\textbf{Solution:} $W_1 \cap W_2 = \{ (0, y, 0) : y \in 
        \mathbb{R} \}$. $V$ is not a direct sum of $W_1$ and $W_2$.
        Also there is no unique sum for every element of $V$. That is
        \begin{align*}
            (1, -2, 1) &= (0, -2, 1) + (1, 0, 0) \\
                       &= (0, -1, 1) + (1, -1, 0) \\
                       & \vdots
        \end{align*}

        \setlength{\fboxsep}{1em}
        \noindent\fbox{
            \begin{minipage}[c]{0.93\linewidth}
                \textbf{Note: }
                \begin{itemize}
                    \item dim($\mathbb{C}^n(\mathbb{R})) = 2n$
                    \item dim($\mathbb{C}^n(\mathbb{C})) = n$
                \end{itemize}
                $$\text{dim($W_1 \cup W_2$)} = \text{dim $W_1$}
                + \text{dim $W_2$} - \text{dim($W_1 \cap W_2$)}$$
                
            \end{minipage}
        }

\section{Linear Transformation}
    \begin{definition}
        Let $V$ and $W$ be two vector spaces over the same field $F$. Then
        $T : v \rightarrow W$ is said to be a linear transformation if 
        \begin{enumerate}[label=(\alph*)]
            \item For all $u, v \in V$, $T(u + v) = T(u) + T(v)$
            \item For all $\alpha \in F$ and $u \in V$, $T(\alpha u) = \alpha T(u)$
        \end{enumerate}
        \begin{center}
            OR
        \end{center}
        For all $\alpha, \beta \in F \text{ and }u, v \in V$, $T(\alpha u + \beta v) = \alpha T(u) + \beta T(v)$    
    \end{definition}
    \textbf{Note: }
    \begin{enumerate}[label=(\roman*)]
        \item If $T(0) \neq 0$, then the map $T$ is not a linear transformation,
        where $0 \in V$ and $W$
    \end{enumerate}

    \subsection{Special Transformations}
        \begin{enumerate}[label=\textbf{\arabic*}]
            \item \textbf{Identity Transformation: }Let $V$ be a vector 
            space over the field $F$, define $I: V \rightarrow V$
            such that \[ T(v) = v \] for all $v \in V$, then $I$ is a linear
            transformation
            \begin{proof}
                Consider $\alpha, \beta \in F$ and $v, u \in I$, then 
                \begin{align*}
                    I(\alpha u + \beta v) &= \alpha u + \beta v \quad 
                    \text{since} \quad \alpha u + \beta v \in V \\
                                        &= \alpha I(u) + \beta I(v)
                \end{align*}
                Therefore $I$ is a linear transformation.
            \end{proof}
            \item \textbf{Zero Transformation: }Let $V, W$ be two subspaces 
            of the vector space $V$ over the field $F$, then the map
            $O: V \rightarrow W$ where $$O(v) = 0$$ is called the 
            linear transformation.
            \begin{proof}
                Consider $\alpha, \beta \in F$ and $v, u \in O$, then 
                \begin{align*}
                    O(\alpha u + \beta v) &= 0 \quad 
                    \text{since} \quad \alpha u + \beta v \in V \\
                    \alpha O(v) + \beta O(u) &= \alpha 0 + \beta 0 \\
                                            &= 0 \\
                    \therefore O(\alpha u + \beta v) &= \alpha u + \beta v
                \end{align*}
                Hence O is a linear transformation.
            \end{proof}
        \end{enumerate}
    
    \subsection{Null Space (Kernel) and Range Space of $T$}
        \begin{definition}
            Let $T$ be a linear transformation from $V$ to $W$, 
            $T: V \rightarrow W$. Then 
            $$ N(T) = \textup{ker($T$)} = \{ x \in V \; | \; T(x) = 0 \} $$
            is called the \textbf{Null Space or Kernel of $T$}
        \end{definition}
        
        \noindent\textbf{Note:} Null space of any linear transformation is always
        non empty because $T(0)$ is always equal to $0$ for any linear 
        transformation $T$.

        \begin{theorem}
            Let $V$ and $U$ be vector spaces over the same field $F$. 
            Let $T: V \rightarrow W$ be a linear transformation, then
            $N(T)$ is a subspace of $V$
        \end{theorem}
        \begin{proof}
            Since $T$ is a linear transformation, $T(0) = 0$, hence
            $0 \in N(T)$ and $N(T) \neq \bm{\phi}$ and also 
            $N(T) \subseteq V$. Now consider $\alpha, \beta \in F$ and 
            $u, v \in N(T)$, then $T(u) = T(v) = 0$.
            \begin{align*}
                T(\alpha u + \beta v) &= \alpha\cdot T(u) + \beta\cdot T(v) \\
                                                  &= \alpha\cdot 0 + \beta\cdot 0 \\
                                                  &= 0
            \end{align*}
            Therefore $\alpha u + \beta v$ also belongs to $N(T)$, which
            implies $N(T)$ is a subspace of $V$
        \end{proof}
        
        \noindent \textbf{Nullity of T: }The dimension of the Null
        Space of any L.T $T$ is called the nullity of $T$, It is 
        represented as $\bm{\eta(T)}$
        
        \noindent \textbf{For a finite dimensional vector space 
        $V$, dim($N(T)$) $\le$ dim($V$)}
        
        \begin{definition}
            Let $V$ and $W$ be finite dimensional vector spaces over the
            same field $F$ and $T:V \to W$ be a linear transformation,
            then
            $$R(T) = \{ w \in W \; | \; \exists \; v \in V, T(v) = w \}$$
            is called the \textbf{Range Space of $T$}
        \end{definition}

        \begin{theorem}
            Let $T: V \to W$ be a linear transformation where $V$ 
            and $W$ are finite dimensional vector spaces over same
            field $F$, then the range space of $T$, $R(T)$ is a 
            subpace of $W$. i.e $R(T) \le W$
        \end{theorem}
        \begin{proof}
            Very simple, similar to null space being a subspace of $V$
        \end{proof}

        \noindent\textbf{Note: }The dimension of range space of $T$
        is denoted as $\rho$($T$) 
        
    \subsection{Linear Transformation Examples}
        \begin{enumerate}[label=\textbf{\arabic*}]
            \item $T: P_4(x) \rightarrow P_3(x)$ and 
            $T\left(p(x)\right) = \displaystyle\int_0^x p(t) dt$
            \item[] The linear transformation is not well defined, all
            polynomials of degree $\ge 3$ lie outside of the codomain
            \item $T: P(x) \rightarrow P(x)$ and 
            $T\left(p(x)\right) = \displaystyle\int_0^x p(t) dt$
            \begin{align*}
                T(\alpha p + \beta q) &= \displaystyle\int_0^x \left(\alpha p(t) + \beta q(t)\right) dt \\
                                    &= \alpha\displaystyle\int_0^x p(t) dt + \beta\displaystyle\int_0^x q(t) dt \\
                                    &= \alpha T(p) + \beta T(q)
            \end{align*}
            \item $T: P(x) \rightarrow P(x)$ and 
            $T\left(p(x)\right) = p''(x)$
            \item[] It is a linear transformation.
            \item $T: P(x) \rightarrow P(x)$ and 
            $T\left(p(x)\right) = p'(x)$
            \item[] It is a linear transformation. 
            \item $T: P(x) \rightarrow P(x)$ and 
            $T\left(p(x)\right) = p''(x) + p(x)$
            \item[] It is a linear transformation. 
        \end{enumerate}
        \begin{theorem}
            Let $T: V \rightarrow W$ be a linear transformation, where
            $V$ and $W$ are vector spaces over a same field $F$. If $S =
            \{ v_1, v_2, \cdots, v_n \}$ be a spanning set of $V$, then the
            set $\{ T(v_1), T(v_2), \cdots, T(v_n) \}$, say $R$ is the
            spanning set of the range space of $T$.
        \end{theorem}
        \begin{proof}
            Range space of $T = \{ w \in W\bigm\vert \exists v \in V, w = 
            T(v) \}$ and since span($S$) $= V$, for every $v 
            \in V$ there exists scalars $c_1, c_2, \cdots, c_n$ from 
            the field $F$ such that 
            \begin{align*}
                v    &= c_1v_1 + c_2v_2 + \cdots + c_nv_n \\
                T(v) &= T(c_1v_1 + c_2v_2 + \cdots + c_nv_n) \\
                w    &= c_1T(v_1) + c_2T(v_2) + \cdots + c_nT(v_n)
            \end{align*}
            Hence $R = \{ T(v_1), T(v_2), \cdots, T(v_n) \}$ spans the
            range space of $T$.
        \end{proof}
        \begin{theorem}
            Let $T: V \rightarrow W$ be a linear transformation, where
            $V$ and $W$ are vector spaces over a same field $F$ and S =
            $\{v_1, v_2, \cdots, v_n\}$ be a subset of $V$. If the
            set $\{T(v_1), T(v_2), \cdots, T(v_n)\}$ is linearly independent
            set, then S is also a linearly independent set.
        \end{theorem}
        \begin{proof}
            Consider the equation \[ c_1v_1 + c_2v_2 + \cdots + c_nv_n = 0 \]
            where $c_i$'s are scalars from the field $F$. Then 
            \begin{align*}
                T(c_1v_1 + c_2v_2 + \cdots + c_nv_n) &= T(0) \\
                c_1T(v_1) + c_2T(v_2) + \cdots + c_nT(v_n) &= 0 
            \end{align*}
            since $\{T(v_1), T(v_2), \cdots, T(v_n)\}$ is an L.I set,
            $c_1 = c_2 = \cdots = c_n = 0$. We just showed that 
            $$c_1v_1 + c_2v_2 + \cdots + c_nv_n = 0 \implies c_1 = c_2 = 
            \cdots = c_n = 0$$. Hence $\{ v_1, v_2, \cdots, v_n \}$ is
            linearly independent.
        \end{proof}
        \begin{theorem}
            Let $V$ and $W$ be two finite dimensional vector spaces over F
            and $\{ v_1, v_2, \cdots, v_n \}$ be a basis for $V$ and 
            $u_1, u_2, \cdots, u_n$ be any vectors in $W$. Then there 
            exists a \textbf{unique} Linear Transformation from $V$ to 
            $W$ such that $T(v_i) = u_i$ for $1 \le i \le n$
        \end{theorem}
        \begin{proof}
            Lenghty, hence skipped.
        \end{proof}
        \begin{theorem} \textup{\textbf{Rank Nullity Theorem}}
            Let $T: V \rightarrow W$ be a linear transformation and $V$
            is a \textbf{finite dimensional vector space}. Then 
            \[ \eta(T) + \rho(T) = \textup{dim }V \]
            \begin{center}
                OR
            \end{center}
            \[ \textup{dim }N(T) + \textup{dim }R(T) = \textup{dim }V \]
        \end{theorem}
        \begin{proof}
            skipped.
        \end{proof}

        \setlength{\fboxsep}{1em}
        \noindent\fbox{
            \begin{minipage}[c]{0.93\linewidth}
                \textbf{Note: }
                \begin{itemize}
                    \item For homogeneous systems dim($R(T)$) = $n - r$
                    \item For non-homogeneous systems dim($R(T)$) = $n - r + 1$
                \end{itemize}
                where $r$ is the rank of the coefficient matrix 
                and $n$ is the number of variables (from the 
                order $m\times n$)
            \end{minipage}
        }

    \subsection{Algebra of Linear Transformation}
        \begin{theorem}
            Let $T: V \rightarrow W and S: V \rightarrow W$ be two linear
            transformation, then their sum $T + S$ and the scalar multiplication
            $cT, c \in F$ are also linear transformations. $V, W, F, c$ have 
            their regular meanings.
        \end{theorem}
        \begin{corollary}
            Let \textup{Hom($V, W$)} be the set of all linear transformations from vector
            spaces $V \rightarrow W$ over the same field $F$, then \textup{Hom($V, W$)} forms
            a subspace of the vector space of all functions or maps.
        \end{corollary}
        If dim($V$) $= m$, dim($W$) $= n$, then dim(Hom($V, W$)) $= mn$
        \begin{theorem}
            Let $T: V \rightarrow W$ and $S: W \rightarrow V$ be two linear
            transformations, then $T \circ S$ and $S \circ T$ are also linear
            transformations from W to W and V to V respectively.
        \end{theorem}
        \begin{proof}
            For all $u, v \in V$ and $\alpha, \beta \in F$ the expression
            \begin{align*}
                (T \circ S)(\alpha u + \beta v) &= T(S(\alpha u + \beta v)) \\
                                                &= T(S(\alpha u) + S(\beta v)) \\
                                                &= T(\alpha S(u) + \beta S(v)) \\
                                                &= T(S(\alpha u)) + T(S(\beta v)) \\
                                                &= \alpha T(S(u)) + \beta T(S(v)) \\
                                                &= \alpha (T \circ S)(u) + \beta (T \circ S)(v)
            \end{align*}
        \end{proof}

        \begin{definition}
            \textbf{Singular Map: }Let $V$ and $W$ be vector subspaces
            over the same field $F$ and $T: V \to W$ be a linear 
            transformation then $T$ is called a singular map if there
            exists a non zero vector $x \in V$ such that $T(x) = 0$
        \end{definition}

        \begin{definition}
            \textbf{Non-Singular Map: }Let $V$ and $W$ be vector subspaces
            over the same field $F$ and $T: V \to W$ be a linear 
            transformation then $T$ is called a singular map if there
            does not exist a non zero vector $x \in V$ such that 
            $T(x) = 0$
        \end{definition}
        \noindent\textbf{Note: }if $\eta(T) > 0$, then $T$ is a singular map
        and if $\eta(T) = 0$ then $T$ is a non-singular map
    \subsection{Invertible Maps}
        \begin{theorem}
            Let $T: V \rightarrow V$ be a linear transformation where $V$
            is a \textbf{finite dimensional vector space}, then
            \begin{align*}
                \textup{ker($T$)} = 0 & \iff \textup{ $T$ is one-one} \\
                                      & \iff \textup{ $T$ is onto} \\
                                      & \iff \textup{ $T$ is bijective} \\
                                      & \iff \textup{ $T$ is invertible} \\
                                      & \iff \textup{ $T$ is non singular}
            \end{align*}
        \end{theorem}
        \setlength{\fboxsep}{1em}
        \noindent\fbox{
            \begin{minipage}[c]{0.93\linewidth}
                \textbf{Note: } For the linear transformation
                $T: V \rightarrow W$
                \begin{enumerate}
                    \item If dim($V$) is finite, then
                    \begin{align*}
                        \textup{ker($T$)} = 0 & \iff \textup{ $T$ is one-one} \\
                                              & \iff \textup{ $T$ is non singular}
                    \end{align*}
                    \item If dim($V$) $=$ dim($W$), then
                    \begin{align*}
                        \textup{ker($T$)} = 0 & \iff \textup{ $T$ is one-one} \\
                                              & \iff \textup{ $T$ is onto} \\
                                              & \iff \textup{ $T$ is bijective} \\
                                              & \iff \textup{ $T$ is invertible} \\
                                              & \iff \textup{ $T$ is non singular}
                    \end{align*}
                \end{enumerate}
            \end{minipage}
        }
    \subsection{Matrix Representation}
        \begin{definition}
            $T: V \rightarrow V$ be a Linear Transformation where $V$
            is a \textbf{finite dimensional vector space}. Let $\{
            v_1, v_2, \cdots, v_n\}$ be a basis of $V$. Then matrix
            representation of $T$ with respect to $B$ is given as $\bm{[T]_B}$
            \begin{align*}
                T(v_1) &= a_{11}v_1 + a_{12}v_2 + \cdots + a_{1n}v_n \\
                T(v_2) &= a_{21}v_1 + a_{22}v_2 + \cdots + a_{2n}v_n \\
                \cdots \\
                T(v_n) &= a_{n1}v_1 + a_{n2}v_2 + \cdots + a_{nn}v_n
            \end{align*}
            $
                \bm{[T]_B} = 
                        \begin{bmatrix}
                            a_{11} & a_{12} & \cdots & a_{1n} \\
                            a_{21} & a_{22} & \cdots & a_{2n} \\
                            \vdots & \vdots & \ddots & \vdots \\
                            a_{n1} & a_{n2} & \cdots & a_{nn}
                        \end{bmatrix} ^ \intercal
            $
        \end{definition}
        $$[T]_{B_1} \sim [T]_{B_2} \sim \cdots [T]_{B_n} \sim \cdots $$
        By similar we mean the matrix representations have \textbf{%
        same eigen values, trace, detereminant} (for a linear 
        operator, i.e a transformation from $V$ to $V$)
    
    \subsection{Change of Basis}
        \begin{definition}
            Let $T: V \to V$ be a linear transformation and $B_1$ 
            and $B_2$ be two bases of $V$. $B_1 = \{ v_1, v_2, 
            \cdots, v_n \}$ and $B_2 = \{ u_1, u_2, \cdots, u_n \}$.
            Then 
            \begin{align*}
                v_1 &= a_{11}u_1 + a_{12}u_2 + \cdots + a_{1n}u_n \\
                v_2 &= a_{21}u_1 + a_{22}u_2 + \cdots + a_{2n}u_n \\
                \vdots \\
                v_n &= a_{n1}u_1 + a_{n2}u_2 + \cdots + a_{nn}u_n
            \end{align*}
            and
            $$
                \bm{[T]_{B_1}^{B_2}} = 
                        \begin{bmatrix}
                            a_{11} & a_{12} & \cdots & a_{1n} \\
                            a_{21} & a_{22} & \cdots & a_{2n} \\
                            \vdots & \vdots & \ddots & \vdots \\
                            a_{n1} & a_{n2} & \cdots & a_{nn}
                        \end{bmatrix} ^ \intercal
            $$ is called the change of basis matrix from $B_2$
            to $B_1$
        \end{definition}

        \noindent \textbf{Basically change of basis gives us a 
        matrix which maps coordinate vectors from one basis to 
        another, that is $\bm{[T]_{B_1}^{B_2}}$ maps coordinate
        vector of vector in $B_2$ basis to coordinate vector of the
        same vector in $B_1$ basis}

        \vspace*{1em}
        \noindent\textbf{Note: }If $P = [T]_{B_1}^{B_2}$ and $Q =
        [T]_{B_2}^{B_1}$, then
        \begin{itemize}
            \item $P$ and $Q$ are invertible matrices
            \item $P$ and $Q$ are inverses of each other
            $$P^{-1} = Q \text{ and } Q^{-1} = P \text{ and }
            PQ = I = QP $$
        \end{itemize}
    
\section{Eigen Values and Eigen Vectors}
    \begin{definition}
        
    \end{definition}
\end{document}