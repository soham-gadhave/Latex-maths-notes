\documentclass[a4paper]{article}
\usepackage[margin=1in]{geometry}
\usepackage[utf8]{inputenc}
\usepackage{amssymb}
\usepackage{enumitem}
\usepackage{nicefrac}
\usepackage{tabularx}
\usepackage{amsmath, mathtools}

\newcounter{examplecounter}
\newcommand\examplenumber{\stepcounter{examplecounter}\textbf{Ex.\arabic{examplecounter}} }

\title{Sequence and Series}
\author{Soham Gadhave}
\date{$4^{\text{th}}$ September 2021}

\begin{document}

\maketitle

\noindent\textbf{Terms of Sequence:} Let $ <a_n> $ be a sequence 
then $ a_1, a_2, \cdots, a_n $ are called terms of a sequence \\ \\ 
\textbf{Range of a Sequence:} The set of all distinct terms of a sequence
is called its range \\ \\
\textbf{Bounded and Unbounded sequence:}
\begin{enumerate}
    \item \textbf{Bounded Above Sequences:} A sequence $<a_n>$ is 
    said to be bounded above if there exists a real number $k$ such 
    that $a_n \le k$ for all $n \in \mathbb{N}$
    \item \textbf{Bounded Below Sequences:} A sequence $<a_n>$ is 
    said to be bounded below if there exists a real number $k$ such 
    that $a_n \ge k$ for all $n \in \mathbb{N}$
    \item[] A sequence which is bounded above and bounded below is called a 
    bounded sequence.
    \item \textbf{Unbounded Sequence: } A sequence is said to be
    unbounded if it is not bounded. 
    \item[\textbf{Ex}] Bound Above and Bound Below sequence: 
    $a_n = <\dfrac{1}{n}>$
    \item[\textbf{Ex}] Neither Bound Above nor Bound Below sequence: 
    $a_n = <(-1)^nn>$
\end{enumerate}
\textbf{Supremum(least-upper bound):} The supremum of the range set
the sequence is called Supremum of that sequence.\\ \\
\textbf{Infimum(greatest-lower bound):} The infimum of the range set
the sequence is called Infimum of that sequence.\\ \\
\textbf{Note:}
\begin{enumerate}
    \item If a sequence is unbounded above then its Supremum is $\infty$
    \item If a sequence is unbounded below then its Infimum is $-\infty$
    \item If for any sequence its Supremum and Infimum are finite (or it 
    exists) then it is a bounded sequence
\end{enumerate}
\textbf{Monotonocity of a Sequence: }
\begin{enumerate}
    \item \textbf{Monotonic Increasing Sequence:} Let $<a_n>$ be a 
    sequence, the this sequence is called monotonically increasing 
    sequence if $a_{n+1} \ge a_n$ for all $n > N, N \in \mathbb{N}$.
    \item \textbf{Strictly Monotonic Increasing Sequence:} Let 
    $<a_n>$ be a sequence, the this sequence is called strictly 
    monotonically increasing sequence if $a_{n+1} \ge a_n$ for all 
    $n \ge N, N \in \mathbb{N}$. 
    \item \textbf{Monotonic Decreasing Sequence:} Let $<a_n>$ be a 
    sequence, the this sequence is called monotonically decreasing 
    sequence if $a_{n+1} \le a_n$ for all $n \ge N, N \in \mathbb{N}$.
    \item \textbf{Strictly Monotonic Decreasing Sequence:} Let 
    $<a_n>$ be a sequence, the this sequence is called strictly 
    monotonically decreasing sequence if $a_{n+1} < a_n$ for all 
    $n \ge N, N \in \mathbb{N}$.
\end{enumerate}
\begin{center}
    \bgroup
    \def\arraystretch{2.5}
        \begin{tabularx}{\linewidth}{|l|X|X|c|}
            \hline
            \textbf{Sequence} & \textbf{Bounded Above} & \textbf{Bounded Below} & \textbf{Range Set} \\
            \hline
            \examplenumber $ a_n = <\dfrac{1}{n}>, 1, \dfrac{1}{2}, \dfrac{1}{3}, \dfrac{1}{4}, \cdots $ & yes, $1$ & yes, $0$ & R $= \{1, \dfrac{1}{2}, \dfrac{1}{3}, \dfrac{1}{4}, \cdots\}$ \\
            \hline
            \examplenumber $ a_n = <(-1)^n>, -1, 1, -1, 1, \cdots $ & yes, $1$ & yes, $-1$ & R $= \{-1, 1\} $ \\
            \hline
            \examplenumber $ a_n = <\dfrac{(-1)^n}{n}>, -1, \dfrac{1}{2}, -\dfrac{1}{3}, \dfrac{1}{4}, \cdots $ & yes, $\dfrac{1}{2}$ & yes, $-1$ & R $= \left\{ -1, \dfrac{1}{2}, -\dfrac{1}{3}, \dfrac{1}{4}, \cdots \right\} $ \\
            \hline
            \examplenumber $ a_n = <1+(-1)^n>, 0, 2, 0, 2, \cdots $ & yes, $0$ & yes, $2$ & R = $\{0, 2\}$\\
            \hline
            \examplenumber $ a_n = <\dfrac{n}{n+1}>, \dfrac{1}{2}, \dfrac{2}{3}, \dfrac{3}{4}, \dfrac{4}{5}, \cdots $ & yes, $1$ & yes, $\dfrac{1}{2}$ & R = $\left\{\dfrac{1}{2}, \dfrac{2}{3}, \dfrac{3}{4}, \dfrac{4}{5}, \cdots\right\}$\\
            \hline
            \examplenumber $ a_n = <n>, 1, 2, 3, 4, \cdots $ & no & yes, $1$ & R = $\{1, 2, 3, 4, \cdots\}$\\
            \hline
            \examplenumber $ a_n = <-n>, -1, -2, -3, -4, \cdots $ & yes, -1 & no & R = $\{-1, -2, -3, -4, \cdots\}$\\
            \hline
            \examplenumber $a_n = 
            \begin{cases*}
                2, \text{if } n \text{ is prime} \\
                n, \text{if } n \text{ is not prime} 
            \end{cases*}
            $ & no & yes, 1 & R $=\{1, 2, 2, 4, 2, 6, \cdots\} $\\
            \hline
            \examplenumber $ a_n = <(-1)^nn>, -1, 2, -3, 4, \cdots $ & no & no & R = $\{-1, 2, -3, 4, \cdots\}$\\
            \hline    
        \end{tabularx}
    \egroup
\end{center}

\pagebreak

\begin{center}
    \date{$5^{\text{th}}$ September 2021} \\
\end{center}
\begin{description}
    \item[\textbf{Limit point / Cluster point}] \hfill \\
    \begin{itemize}
        \item If a sequence is bounded and has only one limit point, then that sequence
        converges to that point.
        \item If a sequence has more than one limit points then its limit
        does not exist. 
        \item \textbf{Examples:}
        \item[] 
            \begin{enumerate}
                \item 
                    $a_n = 
                        \begin{cases}
                           2, \text{if n is prime} \\
                           n, \text{if n is not prime} \\
                        \end{cases}
                    $
                \item[] It has a limit point at 2, because every 
                neighbourhood of 2 has infinite number of terms if
                the sequence. 
            \end{enumerate}
        \item \textbf{Results:}
            \begin{enumerate}
                \item \textbf{Bolzano - Weierstrass Theorem:} Every
                bounded sequence has a limit point.
                \item Unbounded sequence may have a limit point.
            \end{enumerate}
    \end{itemize}
    \item[\textbf{Limit of a Sequence:}] Let $<a_n>$ be a sequence,
    limit of the sequence is denoted by $\displaystyle\lim_{n \to \infty}a_n$.
    \begin{itemize}
        \item \textbf{Result:}
            \begin{enumerate}
                \item A sequence can have atmost one limit.
                \item Unbounded sequence cannot have limit. 
                \item A non-monotonic sequence can have limit. Ex: 
                $<\dfrac{(-1)^n}{n}>$.
                \item A bounded sequence may not have a limit. Ex: $<(-1)^n>$.
                \item Limit of a sequence is also a limit point, 
                but the converse is not true.
            \end{enumerate}
    \end{itemize}
\end{description}

\end{document}